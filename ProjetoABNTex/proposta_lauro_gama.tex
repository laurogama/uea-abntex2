%% abtex2-modelo-projeto-pesquisa.tex, v-1.9.2 laurocesar
%% Copyright 2012-2014 by abnTeX2 group at http://abntex2.googlecode.com/ 
%%
%% This work may be distributed and/or modified under the
%% conditions of the LaTeX Project Public License, either version 1.3
%% of this license or (at your option) any later version.
%% The latest version of this license is in
%%   http://www.latex-project.org/lppl.txt
%% and version 1.3 or later is part of all distributions of LaTeX
%% version 2005/12/01 or later.
%%
%% This work has the LPPL maintenance status `maintained'.
%% 
%% The Current Maintainer of this work is the abnTeX2 team, led
%% by Lauro César Araujo. Further information are available on 
%% http://abntex2.googlecode.com/
%%
%% This work consists of the files abntex2-modelo-projeto-pesquisa.tex
%% and abntex2-modelo-references.bib
%%

% ------------------------------------------------------------------------
% ------------------------------------------------------------------------
% abnTeX2: Modelo de Projeto de pesquisa em conformidade com 
% ABNT NBR 15287:2011 Informação e documentação - Projeto de pesquisa -
% Apresentação 
% ------------------------------------------------------------------------ 
% ------------------------------------------------------------------------

\documentclass[
	% -- opções da classe memoir --
	12pt,				% tamanho da fonte
	openright,			% capítulos começam em pág ímpar (insere página vazia caso preciso)
	oneside,			% para impressão em verso e anverso. Oposto a twoside
	a4paper,			% tamanho do papel.
	% -- opções da classe abntex2 --
	chapter=TITLE,		% títulos de capítulos convetidos em letras maiúsculas 
	%section=TITLE,		% títulos de seções convertidos em letras maiúsculas
	%subsection=TITLE,	% títulos de subseções convertidos em letras maiúsculas
	%subsubsection=TITLE,% títulos de subsubseções convertidos em letras maiúsculas
	% -- opções do pacote babel --
	english,			% idioma adicional para hifenização
	french,				% idioma adicional para hifenização
	spanish,			% idioma adicional para hifenização
	brazil,				% o último idioma é o principal do documento
	article,			% documento divido por sections
	]{uea-abntex2}

%\evensidemargin 0.5 cm

% ---
% PACOTES
% ---

% ---
% Pacotes fundamentais 
% ---
\usepackage{lmodern}			% Usa a fonte Latin Modern
\usepackage[T1]{fontenc}		% Selecao de codigos de fonte.
\usepackage[utf8]{inputenc}		% Codificacao do documento (conversão automática dos acentos)
\usepackage{indentfirst}		% Indenta o primeiro parágrafo de cada seção.
\usepackage{color}				% Controle das cores
\usepackage{graphicx}			% Inclusão de gráficos
\usepackage{microtype} 			% para melhorias de justificação
\usepackage{array}
\usepackage{longtable}
\usepackage{calc}
\usepackage{multirow}
\usepackage{hhline}
\usepackage{ifthen}
\usepackage{lscape}
\usepackage[table]{xcolor}
\def\inputGnumericTable{}
% ---

% ---
% Pacotes adicionais, usados apenas no âmbito do Modelo Canônico do abnteX2
% ---
\usepackage{lipsum}				% para geração de dummy text
% ---

% ---
% Pacotes de citações
% ---
\usepackage[brazilian,hyperpageref]{backref}	 % Paginas com as citações na bibl
\usepackage[num]{abntex2cite}	% Citações padrão ABNT
\usepackage{tocloft}
\usepackage{parskip}

% --- 
% CONFIGURAÇÕES DE PACOTES
% ---

% ---
% Configurações do pacote backref
% Usado sem a opção hyperpageref de backref
\renewcommand{\backrefpagesname}{Citado na(s) página(s):~}
% Texto padrão antes do número das páginas
\renewcommand{\backref}{}
% Define os textos da citação
\renewcommand*{\backrefalt}[4]{
	\ifcase #1 %
		Nenhuma citação no texto.%
	\or
		Citado na página #2.%
	\else
		Citado #1 vezes nas páginas #2.%
	\fi}%
% ---

% ---
% Informações de dados para CAPA e FOLHA DE ROSTO
% ---
\titulo{MONITORAMENTO DE DESCARGAS ELETROESTÁTICAS EM LINHAS DE PRODUÇÃO DE ELETRÔNICOS}
\autor{LAURO MANOEL LIMA DA GAMA}
\local{Manaus}
\data{2014}
\instituicao{%
  UNIVERSIDADE DO ESTADO DO AMAZONAS
  \par
  ESCOLA SUPERIOR DE TECNOLOGIA}
\tipotrabalho{Projeto de Pesquisa}
\orientador[Orientador:]{Paulo Cavalcante}
% O preambulo deve conter o tipo do trabalho, o objetivo, 
% o nome da instituição e a área de concentração 
\preambulo{Projeto de pesquisa proposto durante a disciplina Metodologia da
Pesquisa como pré-requisito para obtenção do título de Especialista em
Desenvolvimento de novos produtos pela Universidade do Estado do Amazonas,
Escola Superior de Tecnologia.}
% ---

% ---
% Configurações de aparência do PDF final

% alterando o aspecto da cor azul
\definecolor{blue}{RGB}{41,5,195}

% informações do PDF
\makeatletter
\hypersetup{
     	%pagebackref=true,
		pdftitle={\@title}, 
		pdfauthor={\@author},
    	pdfsubject={\imprimirpreambulo},
	    pdfcreator={LaTeX with abnTeX2},
		pdfkeywords={esd}{cargas eletroestáticas}{descargas eletroestáticas}, 
		colorlinks=false,       		% false: boxed links; true: colored links
   		linkcolor=black,          	% color of internal links
    	citecolor=blue,        		% color of links to bibliography
    	filecolor=magenta,      		% color of file links
		urlcolor=blue,
		bookmarksdepth=4
}
\makeatother
% --- 

% --- 
% Espaçamentos entre linhas e parágrafos 
% --- 

% O tamanho do parágrafo é dado por:
\setlength{\parindent}{1.3cm}

% Controle do espaçamento entre um parágrafo e outro:
\setlength{\parskip}{0.2cm}  % tente também \onelineskip

% ---
% compila o indice
% ---
\makeindex
% ---

% ----
% Início do documento
% ----
\begin{document}

% Retira espaço extra obsoleto entre as frases.
\frenchspacing 

% ----------------------------------------------------------
% ELEMENTOS PRÉ-TEXTUAIS
% ----------------------------------------------------------
% \pretextual

% ---
% Capa
% ---
\imprimircapa
% ---

% ---
% Folha de rosto
% ---
\imprimirfolhaderosto
% ---

% ---
% NOTA DA ABNT NBR 15287:2011, p. 4:
%  ``Se exigido pela entidade, apresentar os dados curriculares do autor em
%     folha ou página distinta após a folha de rosto.''
% ---

% ---
% inserir lista de ilustrações
% ---
%***********3 linhas comentadas pois não é obrigatória a Lista de Figuras
%\pdfbookmark[0]{\listfigurename}{lof}
%\listoffigures*
%\cleardoublepage
% ---

% ---
% inserir lista de tabelas
% ---
%***********3 linhas comentadas pois não é obrigatória a Lista de Tabelas
%\pdfbookmark[0]{\listtablename}{lot}
%\listoftables*
%\cleardoublepage
% ---

% ---
% inserir lista de abreviaturas e siglas
% ---
\begin{siglas}
  \item[ESD] \textit{Electrostatic Discharge}
  \item[ESDA] \textit{Electrostatic Discharge Association}
  \item[EST] Escola Superior de Tecnologia
  \item[UEA] Universidade do Estado do Amazonas 
\end{siglas}
 ---

% ---
% inserir lista de símbolos
% ---
%\begin{simbolos}
%  \item[$ \Gamma $] Letra grega Gama
%  \item[$ \Lambda $] Lambda
%  \item[$ \zeta $] Letra grega minúscula zeta
%  \item[$ \in $] Pertence
%\end{simbolos}
% ---
% ---
% inserir o sumario
% ---
%\pdfbookmark[0]{\contentsname}{toc}
\renewcommand{\contentsname}{\vspace*{3.4cm}SUMÁRIO}
\tableofcontents*
%\cleardoublepage
% ---

% ----------------------------------------------------------
% ELEMENTOS TEXTUAIS
% ----------------------------------------------------------

% ----------------------------------------------------------
% Introdução
% ----------------------------------------------------------
\textual
\pagestyle{simple}

\newpage

\chapter*{\vspace*{3.4cm}INTRODUÇÃO}
\addcontentsline{toc}{section}{INTRODUÇÃO}
A prevenção de descargas eletroestáticas é uma preocupação constante durante a produção de equipamentos eletrônicos. Tais cargas são responsáveis por falhas e danos aos produtos produzidos e representam prejuízos econômicos consideráveis.

Essas falhas são originadas da degradação ou dano a barreiras dielétricas dos componentes eletrônicos por descargas acima do especificado para uso no componente.

A forma mais eficiente de proteger produtos desse problema é a prevenção de sua ocorrência. Tal prevenção ocorre pelo monitoramento constante de acumulo de cargas geradoras de descargas e a neutralização do agente gerador.
Esse monitoramento pode ser realizado pela constante medição de cargas em diversos pontos e o envio do resultado dessas medições para um sistema computacional que fará a analise e sinalização desses resultados. 

O presente projeto tem como objetivo o desenvolvimento de um sistema de monitoramento de descargas eletroestáticas em linhas de produção de eletrônicos através da medição de cargas eletroestáticas em pessoas e objetos.


% ----------------------------------------------------------
% Capitulo de textual  
% ----------------------------------------------------------
\newpage
%\chapter*{\vspace*{3.4cm}PROJETO DE PESQUISA}

\vspace{24pt}
\section{TEMA}
DESCARGAS ELÉTRICAS
\section{DELIMITAÇÃO DO TEMA}
Monitoramento de descargas eletroestáticas em linhas de produção de eletrônicos.
\section{FORMULAÇÃO DO PROBLEMA}
A carência de um sistema que monitore cargas eletroestáticas durante o processo de produção afim de evitar descargas eletroestáticas que possam danificar componentes eletrônicos.  
\section{HIPÓTESE}
É possível a criação de um equipamento que meça a quantidade de cargas estáticas em pessoas e equipamentos envolvidos no processo de manufatura de eletroeletrônicos e dissipe essas cargas afim de prevenir a ocorrência de descargas eletroestáticas utilizando módulos microcontroladores e transmissão via rede de radio \textit{wifi}. 
\section{OBJETIVO}
Estudar a arquitetura de um sistema que monitore as cargas estáticas em pessoas e equipamentos envolvidos no processo de manufatura de eletroeletrônicos.

Projetar um protótipo do sistema que monitore postos de trabalho onde o controle de cargas eletroestáticas seja necessário, indicando quando o objeto sob analise está acumulando cargas estáticas e transmita essa informação via rede de dados \textit{wifi} a um servidor de dados que sinalize e armazene essas informações para posterior analise.

Serão utilizados para o desenvolvimento do protótipo as instalações e recursos próprios do pesquisador.
\section{JUSTIFICATIVA}

\subsection{Justificativa Acadêmica}
O estudo do monitoramento de cargas eletroestáticas permite o aprofundamento da pesquisa de monitoramento remoto, eletrônica e sistemas embarcados.
\subsection{Justificativa Social}
Sistemas de monitoramento podem ser utilizados nos mais diversos processos produtivos para assegurar normas de segurança e aumentar a eficiência da linha.
A utilização desse sistema irá diminuir a incidência de descargas eletroestáticas durante o processo de manufatura diminuindo perdas econômicas e logísticas.
\section{REFERENCIAL TEÓRICO}

\subsection{DESCARGA ELETROESTÁTICA}
A descarga eletrostática (ESD, do inglês \textit{electrostatic
discharge}) é um fenômeno natural que pode ser definido como "a rápida e espontânea transferência de carga entre dois corpos em diferentes potenciais elétricos".\cite{esda}

A descarga eletrostática pode pode causar grandes danos a equipamentos eletrônicos \cite{Katsivelis2010} e segundo \citeauthor{Hwang2005}, um terço das falhas em campo de circuitos integrados são decorrentes de ESD e outras falhas conhecidas como sobrecargas elétricas.

Existem 3 principais processos de geração de cargas eletrostáticas:
\subsubsection{carregamento triboelétrico}
Causado pela fricção de diferentes materiais e é o método mais usual de geração de cargas eletroestáticas.

O carregamento triboelétrico é causado por um principio de contato e separação dos materiais. Quando dois materiais com propriedades triboelétricas diferentes são colocados em contato e separados, elétrons carregados negativamente são transferidos da superfície de um material para o outro. O material que sofre perda e o que ganha elétrons é definido por suas propriedades triboelétricas\cite{Hwang2005}. 

\subsubsection{indução}
O processo de indução ocorre quando um objeto condutor mas sem cargas é colocado próximo a um objeto com cargas eletroestáticas. Ao afastar os objetos aquele que não possuía cargas passa a possuir uma carga resultante da somatória algébrica das cargas. A nova carga possui polaridade oposta a do objeto que foi aproximado.
\subsubsection{condução} 
A condução ocorre quando há contato entre objetos com diferentes potenciais de tensão. Ao entrarem em contato os objetos irão se balancear eletricamente resultando em objetos com cargas com polaridades iguais.

\subsection{DANOS PROVOCADOS POR DESCARGAS ELETROESTÁTICAS}
As descargas eletroestáticas podem causar riscos tanto as pessoas quanto a bens e equipamentos.
Em industrias que lidam com substancias inflamáveis as descargas eletroestáticas podem gerar faíscas e inflamar misturas explosivas. \cite{kassebaum}

De acordo com \citeauthor{Hwang2005}, 58\% das falhas de circuitos integrados baseados em silício e 27\% dos baseados em Gálio-arsênico são decorrentes de ESD e sobrecargas elétricas.

É estimado que as perdas da industria de eletrônicos com descargas eletroestáticas seja de bilhões anualmente. \cite{Hwang2005}

Os custos podem variar de alguns centavos para centenas de reais por componente.

\subsection{MEDIÇÃO DE CARGAS ELETROESTÁTICAS}
A medição de cargas eletroestáticas pode ser feita através de medidores de tensão que medem a carga em um componente de um componente ou medindo sua curva de descarga.\cite{Berndt2010}

Não existem níveis de aceitação padrão, sendo estes dependentes de cada componente a ser testado.  
\section{METODOLOGIA}

\hspace*{0.8cm}Serão feitas pesquisas bibliográficas na área de sistemas microprocessados, com foco na arquitetura Arduino, com foco na linguagem C e programação orientada a objetos com foco na linguagem Python\cite{python} que auxilia a criação do servidor de dados e interface web necessária a visualização dos dados. Serão, por fim, feitas pesquisas sobre circuitos de medição de tensão elétrica, aplicados na leitura de sinais dos objetos a serem testados.

Pesquisas de campo serão aplicadas para coletar dados reais em linhas de produção, serão feitas, também, simulações computacionais e reais nas quais se buscará avaliar a confiabilidade dos algoritmos testados, bem como determinar o mais adequado às limitações inerentes à plataforma de trabalho disponível.

A construção do sistema será dividida em três etapas: A primeira etapa será a implementação dos algoritmos de cálculo de medição de cargas eletroestáticas em \textit{Python} utilizando o pacote \textit{Numpy} de analise matemática.\cite{numpy}

A segunda etapa será a implementação de um protótipo utilizando o modulo arduino Uno como base para o esquema eletrônico. O algorítimo de medição desenvolvido na etapa um será transcrito para a linguagem C e embarcado no modulo de medição.  \cite{arduino}

A terceira etapa sera o desenvolvimento de um programa servidor de dados em linguagem Python utilizando os pacotes Django\cite{django} e \textit{Cherry Py}\cite{cherrypy}.

Após a construção do sistema ele será testado em ambientes laboratoriais e apos essa etapa de testes e subsequente correções, serão feitos testes de campo em uma linha de produção.


\newpage

\section{CRONOGRAMA}

As atividades de desenvolvimento do projeto seguirão o seguinte cronograma:
\newpage

\begin{landscape}

%%%%%%%%%%%%%%%%%%%%%%%%%%%%%%%%%%%%%%%%%%%%%%%%%%%%%%%%%%%%%%%%%%%%%%
%%                                                                  %%
%%  This is the header of a LaTeX2e file exported from Gnumeric.    %%
%%                                                                  %%
%%  This file can be compiled as it stands or included in another   %%
%%  LaTeX document. The table is based on the longtable package so  %%
%%  the longtable options (headers, footers...) can be set in the   %%
%%  preamble section below (see PRAMBLE).                           %%
%%                                                                  %%
%%  To include the file in another, the following two lines must be %%
%%  in the including file:                                          %%
%%        \def\inputGnumericTable{}                                 %%
%%  at the beginning of the file and:                               %%
%%        \input{name-of-this-file.tex}                             %%
%%  where the table is to be placed. Note also that the including   %%
%%  file must use the following packages for the table to be        %%
%%  rendered correctly:                                             %%
%%    \usepackage[utf8]{inputenc}                                 %%
%%    \usepackage{color}                                            %%
%%    \usepackage{array}                                            %%
%%    \usepackage{longtable}                                        %%
%%    \usepackage{calc}                                             %%
%%    \usepackage{multirow}                                         %%
%%    \usepackage{hhline}                                           %%
%%    \usepackage{ifthen}											%%
%%	  \usepackage[table]{xcolor}                                    %%
%%  optionally (for landscape tables embedded in another document): %%
%%    \usepackage{lscape}                                           %%
%%                                                                  %%
%%%%%%%%%%%%%%%%%%%%%%%%%%%%%%%%%%%%%%%%%%%%%%%%%%%%%%%%%%%%%%%%%%%%%%



%%  This section checks if we are begin input into another file or  %%
%%  the file will be compiled alone. First use a macro taken from   %%
%%  the TeXbook ex 7.7 (suggestion of Han-Wen Nienhuys).            %%
\def\ifundefined#1{\expandafter\ifx\csname#1\endcsname\relax}


%%  Check for the \def token for inputed files. If it is not        %%
%%  defined, the file will be processed as a standalone and the     %%
%%  preamble will be used.                                          %%
\ifundefined{inputGnumericTable}

%%  We must be able to close or not the document at the end.        %%
	\def\gnumericTableEnd{\end{document}}


%%%%%%%%%%%%%%%%%%%%%%%%%%%%%%%%%%%%%%%%%%%%%%%%%%%%%%%%%%%%%%%%%%%%%%
%%                                                                  %%
%%  This is the PREAMBLE. Change these values to get the right      %%
%%  paper size and other niceties. Uncomment the landscape option   %%
%%  to the documentclass defintion for standalone documents.        %%
%%                                                                  %%
%%%%%%%%%%%%%%%%%%%%%%%%%%%%%%%%%%%%%%%%%%%%%%%%%%%%%%%%%%%%%%%%%%%%%%

	\documentclass[12pt%
			  ,landscape%
                    ]{extarticle}
       \usepackage[utf8]{inputenc}
	\usepackage{fullpage}
	\usepackage{color}
       \usepackage{array}
	\usepackage{longtable}
       \usepackage{calc}
       \usepackage{multirow}
       \usepackage{hhline}
       \usepackage{ifthen}
       \usepackage[table]{xcolor}

	\begin{document}


%%  End of the preamble for the standalone. The next section is for %%
%%  documents which are included into other LaTeX2e files.          %%
\else

%%  We are not a stand alone document. For a regular table, we will %%
%%  have no preamble and only define the closing to mean nothing.   %%
    \def\gnumericTableEnd{}

%%  If we want landscape mode in an embedded document, comment out  %%
%%  the line above and uncomment the two below. The table will      %%
%%  begin on a new page and run in landscape mode.                  %%
%       \def\gnumericTableEnd{\end{landscape}}
%       \begin{landscape}


%%  End of the else clause for this file being \input.              %%
\fi

%%%%%%%%%%%%%%%%%%%%%%%%%%%%%%%%%%%%%%%%%%%%%%%%%%%%%%%%%%%%%%%%%%%%%%
%%                                                                  %%
%%  The rest is the gnumeric table, except for the closing          %%
%%  statement. Changes below will alter the table's appearance.     %%
%%                                                                  %%
%%%%%%%%%%%%%%%%%%%%%%%%%%%%%%%%%%%%%%%%%%%%%%%%%%%%%%%%%%%%%%%%%%%%%%
\providecommand{\gnumericmathit}[1]{#1} 
%%  Uncomment the next line if you would like your numbers to be in %%
%%  italics if they are italizised in the gnumeric table.           %%
%\renewcommand{\gnumericmathit}[1]{\mathit{#1}}
\providecommand{\gnumericPB}[1]%
{\let\gnumericTemp=\\#1\let\\=\gnumericTemp\hspace{0pt}}
 \ifundefined{gnumericTableWidthDefined}
        \newlength{\gnumericTableWidth}
        \newlength{\gnumericTableWidthComplete}
        \newlength{\gnumericMultiRowLength}
        \global\def\gnumericTableWidthDefined{}
 \fi
%% The following setting protects this code from babel shorthands.  %%
 \ifthenelse{\isundefined{\languageshorthands}}{}{\languageshorthands{english}}
%%  The default table format retains the relative column widths of  %%
%%  gnumeric. They can easily be changed to c, r or l. In that case %%
%%  you may want to comment out the next line and uncomment the one %%
%%  thereafter                                                      %%
\providecommand\gnumbox{\makebox[0pt]}
%%\providecommand\gnumbox[1][]{\makebox}

%% to adjust positions in multirow situations                       %%
\setlength{\bigstrutjot}{\jot}
\setlength{\extrarowheight}{\doublerulesep}

%%  The \setlongtables command keeps column widths the same across  %%
%%  pages. Simply comment out next line for varying column widths.  %%
\setlongtables

\setlength\gnumericTableWidth{%
	100pt+%
	10pt+%
	10pt+%
	10pt+%
	10pt+%
	10pt+%
	10pt+%
	10pt+%
	10pt+%
	10pt+%
	10pt+%
	10pt+%
	10pt+%
	10pt+%
	10pt+%
	10pt+%
	10pt+%
	10pt+%
	10pt+%
	10pt+%
	10pt+%
	10pt+%
	10pt+%
	10pt+%
	10pt+%
	10pt+%
	10pt+%
	10pt+%
	10pt+%
	10pt+%
	10pt+%
	10pt+%
	10pt+%
	10pt+%
	10pt+%
	10pt+%
	10pt+%
0pt}
\def\gumericNumCols{38}
\setlength\gnumericTableWidthComplete{\gnumericTableWidth+%
         \tabcolsep*\gumericNumCols*2+\arrayrulewidth*\gumericNumCols}
%%\ifthenelse{\lengthtest{\gnumericTableWidthComplete > \linewidth}}%
%%         {\def\gnumericScale{\ratio{\linewidth-%
%%                        \tabcolsep*\gumericNumCols*2-%
%%                        \arrayrulewidth*\gumericNumCols}%
%%{\gnumericTableWidth}}}%
%%{\def\gnumericScale{1}}

\def\gnumericScale{1}

%%%%%%%%%%%%%%%%%%%%%%%%%%%%%%%%%%%%%%%%%%%%%%%%%%%%%%%%%%%%%%%%%%%%%%
%%                                                                  %%
%% The following are the widths of the various columns. We are      %%
%% defining them here because then they are easier to change.       %%
%% Depending on the cell formats we may use them more than once.    %%
%%                                                                  %%
%%%%%%%%%%%%%%%%%%%%%%%%%%%%%%%%%%%%%%%%%%%%%%%%%%%%%%%%%%%%%%%%%%%%%%

\ifthenelse{\isundefined{\gnumericColA}}{\newlength{\gnumericColA}}{}\settowidth{\gnumericColA}{\begin{tabular}{@{}p{60pt*\gnumericScale}@{}}x\end{tabular}}
\ifthenelse{\isundefined{\gnumericColB}}{\newlength{\gnumericColB}}{}\settowidth{\gnumericColB}{\begin{tabular}{@{}p{4pt*\gnumericScale}@{}}x\end{tabular}}
\ifthenelse{\isundefined{\gnumericColC}}{\newlength{\gnumericColC}}{}\settowidth{\gnumericColC}{\begin{tabular}{@{}p{4pt*\gnumericScale}@{}}x\end{tabular}}
\ifthenelse{\isundefined{\gnumericColD}}{\newlength{\gnumericColD}}{}\settowidth{\gnumericColD}{\begin{tabular}{@{}p{4pt*\gnumericScale}@{}}x\end{tabular}}
\ifthenelse{\isundefined{\gnumericColE}}{\newlength{\gnumericColE}}{}\settowidth{\gnumericColE}{\begin{tabular}{@{}p{4pt*\gnumericScale}@{}}x\end{tabular}}
\ifthenelse{\isundefined{\gnumericColF}}{\newlength{\gnumericColF}}{}\settowidth{\gnumericColF}{\begin{tabular}{@{}p{4pt*\gnumericScale}@{}}x\end{tabular}}
\ifthenelse{\isundefined{\gnumericColG}}{\newlength{\gnumericColG}}{}\settowidth{\gnumericColG}{\begin{tabular}{@{}p{4pt*\gnumericScale}@{}}x\end{tabular}}
\ifthenelse{\isundefined{\gnumericColH}}{\newlength{\gnumericColH}}{}\settowidth{\gnumericColH}{\begin{tabular}{@{}p{4pt*\gnumericScale}@{}}x\end{tabular}}
\ifthenelse{\isundefined{\gnumericColI}}{\newlength{\gnumericColI}}{}\settowidth{\gnumericColI}{\begin{tabular}{@{}p{4pt*\gnumericScale}@{}}x\end{tabular}}
\ifthenelse{\isundefined{\gnumericColJ}}{\newlength{\gnumericColJ}}{}\settowidth{\gnumericColJ}{\begin{tabular}{@{}p{4pt*\gnumericScale}@{}}x\end{tabular}}
\ifthenelse{\isundefined{\gnumericColK}}{\newlength{\gnumericColK}}{}\settowidth{\gnumericColK}{\begin{tabular}{@{}p{4pt*\gnumericScale}@{}}x\end{tabular}}
\ifthenelse{\isundefined{\gnumericColL}}{\newlength{\gnumericColL}}{}\settowidth{\gnumericColL}{\begin{tabular}{@{}p{4pt*\gnumericScale}@{}}x\end{tabular}}
\ifthenelse{\isundefined{\gnumericColM}}{\newlength{\gnumericColM}}{}\settowidth{\gnumericColM}{\begin{tabular}{@{}p{4pt*\gnumericScale}@{}}x\end{tabular}}
\ifthenelse{\isundefined{\gnumericColN}}{\newlength{\gnumericColN}}{}\settowidth{\gnumericColN}{\begin{tabular}{@{}p{4pt*\gnumericScale}@{}}x\end{tabular}}
\ifthenelse{\isundefined{\gnumericColO}}{\newlength{\gnumericColO}}{}\settowidth{\gnumericColO}{\begin{tabular}{@{}p{4pt*\gnumericScale}@{}}x\end{tabular}}
\ifthenelse{\isundefined{\gnumericColP}}{\newlength{\gnumericColP}}{}\settowidth{\gnumericColP}{\begin{tabular}{@{}p{4pt*\gnumericScale}@{}}x\end{tabular}}
\ifthenelse{\isundefined{\gnumericColQ}}{\newlength{\gnumericColQ}}{}\settowidth{\gnumericColQ}{\begin{tabular}{@{}p{4pt*\gnumericScale}@{}}x\end{tabular}}
\ifthenelse{\isundefined{\gnumericColR}}{\newlength{\gnumericColR}}{}\settowidth{\gnumericColR}{\begin{tabular}{@{}p{4pt*\gnumericScale}@{}}x\end{tabular}}
\ifthenelse{\isundefined{\gnumericColS}}{\newlength{\gnumericColS}}{}\settowidth{\gnumericColS}{\begin{tabular}{@{}p{4pt*\gnumericScale}@{}}x\end{tabular}}
\ifthenelse{\isundefined{\gnumericColT}}{\newlength{\gnumericColT}}{}\settowidth{\gnumericColT}{\begin{tabular}{@{}p{4pt*\gnumericScale}@{}}x\end{tabular}}
\ifthenelse{\isundefined{\gnumericColU}}{\newlength{\gnumericColU}}{}\settowidth{\gnumericColU}{\begin{tabular}{@{}p{4pt*\gnumericScale}@{}}x\end{tabular}}
\ifthenelse{\isundefined{\gnumericColV}}{\newlength{\gnumericColV}}{}\settowidth{\gnumericColV}{\begin{tabular}{@{}p{4pt*\gnumericScale}@{}}x\end{tabular}}
\ifthenelse{\isundefined{\gnumericColW}}{\newlength{\gnumericColW}}{}\settowidth{\gnumericColW}{\begin{tabular}{@{}p{4pt*\gnumericScale}@{}}x\end{tabular}}
\ifthenelse{\isundefined{\gnumericColX}}{\newlength{\gnumericColX}}{}\settowidth{\gnumericColX}{\begin{tabular}{@{}p{4pt*\gnumericScale}@{}}x\end{tabular}}
\ifthenelse{\isundefined{\gnumericColY}}{\newlength{\gnumericColY}}{}\settowidth{\gnumericColY}{\begin{tabular}{@{}p{4pt*\gnumericScale}@{}}x\end{tabular}}
\ifthenelse{\isundefined{\gnumericColZ}}{\newlength{\gnumericColZ}}{}\settowidth{\gnumericColZ}{\begin{tabular}{@{}p{4pt*\gnumericScale}@{}}x\end{tabular}}
\ifthenelse{\isundefined{\gnumericColAA}}{\newlength{\gnumericColAA}}{}\settowidth{\gnumericColAA}{\begin{tabular}{@{}p{4pt*\gnumericScale}@{}}x\end{tabular}}
\ifthenelse{\isundefined{\gnumericColAB}}{\newlength{\gnumericColAB}}{}\settowidth{\gnumericColAB}{\begin{tabular}{@{}p{4pt*\gnumericScale}@{}}x\end{tabular}}
\ifthenelse{\isundefined{\gnumericColAC}}{\newlength{\gnumericColAC}}{}\settowidth{\gnumericColAC}{\begin{tabular}{@{}p{4pt*\gnumericScale}@{}}x\end{tabular}}
\ifthenelse{\isundefined{\gnumericColAD}}{\newlength{\gnumericColAD}}{}\settowidth{\gnumericColAD}{\begin{tabular}{@{}p{4pt*\gnumericScale}@{}}x\end{tabular}}
\ifthenelse{\isundefined{\gnumericColAE}}{\newlength{\gnumericColAE}}{}\settowidth{\gnumericColAE}{\begin{tabular}{@{}p{4pt*\gnumericScale}@{}}x\end{tabular}}
\ifthenelse{\isundefined{\gnumericColAF}}{\newlength{\gnumericColAF}}{}\settowidth{\gnumericColAF}{\begin{tabular}{@{}p{4pt*\gnumericScale}@{}}x\end{tabular}}
\ifthenelse{\isundefined{\gnumericColAG}}{\newlength{\gnumericColAG}}{}\settowidth{\gnumericColAG}{\begin{tabular}{@{}p{4pt*\gnumericScale}@{}}x\end{tabular}}
\ifthenelse{\isundefined{\gnumericColAH}}{\newlength{\gnumericColAH}}{}\settowidth{\gnumericColAH}{\begin{tabular}{@{}p{4pt*\gnumericScale}@{}}x\end{tabular}}
\ifthenelse{\isundefined{\gnumericColAI}}{\newlength{\gnumericColAI}}{}\settowidth{\gnumericColAI}{\begin{tabular}{@{}p{4pt*\gnumericScale}@{}}x\end{tabular}}
\ifthenelse{\isundefined{\gnumericColAJ}}{\newlength{\gnumericColAJ}}{}\settowidth{\gnumericColAJ}{\begin{tabular}{@{}p{4pt*\gnumericScale}@{}}x\end{tabular}}
\ifthenelse{\isundefined{\gnumericColAK}}{\newlength{\gnumericColAK}}{}\settowidth{\gnumericColAK}{\begin{tabular}{@{}p{4pt*\gnumericScale}@{}}x\end{tabular}}

{
\begingroup
\fontsize{5pt}{6pt}\selectfont

\begin{longtable}[c]{%
	b{\gnumericColA}%
	b{\gnumericColB}%
	b{\gnumericColC}%
	b{\gnumericColD}%
	b{\gnumericColE}%
	b{\gnumericColF}%
	b{\gnumericColG}%
	b{\gnumericColH}%
	b{\gnumericColI}%
	b{\gnumericColJ}%
	b{\gnumericColK}%
	b{\gnumericColL}%
	b{\gnumericColM}%
	b{\gnumericColN}%
	b{\gnumericColO}%
	b{\gnumericColP}%
	b{\gnumericColQ}%
	b{\gnumericColR}%
	b{\gnumericColS}%
	b{\gnumericColT}%
	b{\gnumericColU}%
	b{\gnumericColV}%
	b{\gnumericColW}%
	b{\gnumericColX}%
	b{\gnumericColY}%
	b{\gnumericColZ}%
	b{\gnumericColAA}%
	b{\gnumericColAB}%
	b{\gnumericColAC}%
	b{\gnumericColAD}%
	b{\gnumericColAE}%
	b{\gnumericColAF}%
	b{\gnumericColAG}%
	b{\gnumericColAH}%
	b{\gnumericColAI}%
	b{\gnumericColAJ}%
	b{\gnumericColAK}%
	}

\caption{Cronograma de atividades}
\label{tab:cronograma}
\\

%%%%%%%%%%%%%%%%%%%%%%%%%%%%%%%%%%%%%%%%%%%%%%%%%%%%%%%%%%%%%%%%%%%%%%
%%  The longtable options. (Caption, headers... see Goosens, p.124) %%
%	\caption{The Table Caption.}             \\	%
% \hline	% Across the top of the table.
%%  The rest of these options are table rows which are placed on    %%
%%  the first, last or every page. Use \multicolumn if you want.    %%

%%  Header for the first page.                                      %%
%	\multicolumn{38}{c}{The First Header} \\ \hline 
%	\multicolumn{1}{c}{colTag}	%Column 1
%	&\multicolumn{1}{c}{colTag}	%Column 2
%	&\multicolumn{1}{c}{colTag}	%Column 3
%	&\multicolumn{1}{c}{colTag}	%Column 4
%	&\multicolumn{1}{c}{colTag}	%Column 5
%	&\multicolumn{1}{c}{colTag}	%Column 6
%	&\multicolumn{1}{c}{colTag}	%Column 7
%	&\multicolumn{1}{c}{colTag}	%Column 8
%	&\multicolumn{1}{c}{colTag}	%Column 9
%	&\multicolumn{1}{c}{colTag}	%Column 10
%	&\multicolumn{1}{c}{colTag}	%Column 11
%	&\multicolumn{1}{c}{colTag}	%Column 12
%	&\multicolumn{1}{c}{colTag}	%Column 13
%	&\multicolumn{1}{c}{colTag}	%Column 14
%	&\multicolumn{1}{c}{colTag}	%Column 15
%	&\multicolumn{1}{c}{colTag}	%Column 16
%	&\multicolumn{1}{c}{colTag}	%Column 17
%	&\multicolumn{1}{c}{colTag}	%Column 18
%	&\multicolumn{1}{c}{colTag}	%Column 19
%	&\multicolumn{1}{c}{colTag}	%Column 20
%	&\multicolumn{1}{c}{colTag}	%Column 21
%	&\multicolumn{1}{c}{colTag}	%Column 22
%	&\multicolumn{1}{c}{colTag}	%Column 23
%	&\multicolumn{1}{c}{colTag}	%Column 24
%	&\multicolumn{1}{c}{colTag}	%Column 25
%	&\multicolumn{1}{c}{colTag}	%Column 26
%	&\multicolumn{1}{c}{colTag}	%Column 27
%	&\multicolumn{1}{c}{colTag}	%Column 28
%	&\multicolumn{1}{c}{colTag}	%Column 29
%	&\multicolumn{1}{c}{colTag}	%Column 30
%	&\multicolumn{1}{c}{colTag}	%Column 31
%	&\multicolumn{1}{c}{colTag}	%Column 32
%	&\multicolumn{1}{c}{colTag}	%Column 33
%	&\multicolumn{1}{c}{colTag}	%Column 34
%	&\multicolumn{1}{c}{colTag}	%Column 35
%	&\multicolumn{1}{c}{colTag}	%Column 36
%	&\multicolumn{1}{c}{colTag}	%Column 37
%	&\multicolumn{1}{c}{colTag}	\\ \hline %Last column
%	\endfirsthead

%%  The running header definition.                                  %%
%	\hline
%	\multicolumn{38}{l}{\ldots\small\slshape continued} \\ \hline
%	\multicolumn{1}{c}{colTag}	%Column 1
%	&\multicolumn{1}{c}{colTag}	%Column 2
%	&\multicolumn{1}{c}{colTag}	%Column 3
%	&\multicolumn{1}{c}{colTag}	%Column 4
%	&\multicolumn{1}{c}{colTag}	%Column 5
%	&\multicolumn{1}{c}{colTag}	%Column 6
%	&\multicolumn{1}{c}{colTag}	%Column 7
%	&\multicolumn{1}{c}{colTag}	%Column 8
%	&\multicolumn{1}{c}{colTag}	%Column 9
%	&\multicolumn{1}{c}{colTag}	%Column 10
%	&\multicolumn{1}{c}{colTag}	%Column 11
%	&\multicolumn{1}{c}{colTag}	%Column 12
%	&\multicolumn{1}{c}{colTag}	%Column 13
%	&\multicolumn{1}{c}{colTag}	%Column 14
%	&\multicolumn{1}{c}{colTag}	%Column 15
%	&\multicolumn{1}{c}{colTag}	%Column 16
%	&\multicolumn{1}{c}{colTag}	%Column 17
%	&\multicolumn{1}{c}{colTag}	%Column 18
%	&\multicolumn{1}{c}{colTag}	%Column 19
%	&\multicolumn{1}{c}{colTag}	%Column 20
%	&\multicolumn{1}{c}{colTag}	%Column 21
%	&\multicolumn{1}{c}{colTag}	%Column 22
%	&\multicolumn{1}{c}{colTag}	%Column 23
%	&\multicolumn{1}{c}{colTag}	%Column 24
%	&\multicolumn{1}{c}{colTag}	%Column 25
%	&\multicolumn{1}{c}{colTag}	%Column 26
%	&\multicolumn{1}{c}{colTag}	%Column 27
%	&\multicolumn{1}{c}{colTag}	%Column 28
%	&\multicolumn{1}{c}{colTag}	%Column 29
%	&\multicolumn{1}{c}{colTag}	%Column 30
%	&\multicolumn{1}{c}{colTag}	%Column 31
%	&\multicolumn{1}{c}{colTag}	%Column 32
%	&\multicolumn{1}{c}{colTag}	%Column 33
%	&\multicolumn{1}{c}{colTag}	%Column 34
%	&\multicolumn{1}{c}{colTag}	%Column 35
%	&\multicolumn{1}{c}{colTag}	%Column 36
%	&\multicolumn{1}{c}{colTag}	%Column 37
%	&\multicolumn{1}{c}{colTag}	\\ \hline %Last column
%	\endhead

%%  The running footer definition.                                  %%
%	\hline
%	\multicolumn{38}{r}{\small\slshape continued\ldots} \\
%	\endfoot

%%  The ending footer definition.                                   %%
%	\multicolumn{38}{c}{That's all folks} \\ \hline 
%	\endlastfoot
%%%%%%%%%%%%%%%%%%%%%%%%%%%%%%%%%%%%%%%%%%%%%%%%%%%%%%%%%%%%%%%%%%%%%%

\hhline{|-|-|-|-|-|-|-|-|-|-|-|-|-|-|-|-|-|-|-|-|-|-|-|-|-|-|-|-|-|-|-|-|-|-|-|-|-}
\multicolumn{1}{|p{\gnumericColA}|}%
	{\setlength{\gnumericMultiRowLength}{0pt}%
	 \addtolength{\gnumericMultiRowLength}{\gnumericColA}%
	 \multirow{3}[1]{\gnumericMultiRowLength}{\parbox{\gnumericMultiRowLength}{%
	 \gnumericPB{\centering}\textbf{Atividade}}}}
	&\multicolumn{16}{p{	\gnumericColB+%
	\gnumericColC+%
	\gnumericColD+%
	\gnumericColE+%
	\gnumericColF+%
	\gnumericColG+%
	\gnumericColH+%
	\gnumericColI+%
	\gnumericColJ+%
	\gnumericColK+%
	\gnumericColL+%
	\gnumericColM+%
	\gnumericColN+%
	\gnumericColO+%
	\gnumericColP+%
	\gnumericColQ+%
	\tabcolsep*2*15}|}%
	{\gnumericPB{\centering}\textbf{2014}}
	&\multicolumn{20}{p{	\gnumericColR+%
	\gnumericColS+%
	\gnumericColT+%
	\gnumericColU+%
	\gnumericColV+%
	\gnumericColW+%
	\gnumericColX+%
	\gnumericColY+%
	\gnumericColZ+%
	\gnumericColAA+%
	\gnumericColAB+%
	\gnumericColAC+%
	\gnumericColAD+%
	\gnumericColAE+%
	\gnumericColAF+%
	\gnumericColAG+%
	\gnumericColAH+%
	\gnumericColAI+%
	\gnumericColAJ+%
	\gnumericColAK+%
	\tabcolsep*2*19}|}%
	{\gnumericPB{\centering}\textbf{2015}}
\\
\hhline{|~------------------------------------|}
	\multicolumn{1}{|p{\gnumericColA}|}%
	{}
	&\multicolumn{4}{p{	\gnumericColB+%
	\gnumericColC+%
	\gnumericColD+%
	\gnumericColE+%
	\tabcolsep*2*3}|}%
	{\gnumericPB{\centering}\textbf{SET}}
	&\multicolumn{4}{p{	\gnumericColF+%
	\gnumericColG+%
	\gnumericColH+%
	\gnumericColI+%
	\tabcolsep*2*3}|}%
	{\gnumericPB{\centering}\textbf{OUT}}
	&\multicolumn{4}{p{	\gnumericColJ+%
	\gnumericColK+%
	\gnumericColL+%
	\gnumericColM+%
	\tabcolsep*2*3}|}%
	{\gnumericPB{\centering}\textbf{NOV}}
	&\multicolumn{4}{p{	\gnumericColN+%
	\gnumericColO+%
	\gnumericColP+%
	\gnumericColQ+%
	\tabcolsep*2*3}|}%
	{\gnumericPB{\centering}\textbf{DEZ}}
	&\multicolumn{4}{p{	\gnumericColR+%
	\gnumericColS+%
	\gnumericColT+%
	\gnumericColU+%
	\tabcolsep*2*3}|}%
	{\gnumericPB{\centering}\textbf{JAN}}
	&\multicolumn{4}{p{	\gnumericColV+%
	\gnumericColW+%
	\gnumericColX+%
	\gnumericColY+%
	\tabcolsep*2*3}|}%
	{\gnumericPB{\centering}\textbf{FEV}}
	&\multicolumn{4}{p{	\gnumericColZ+%
	\gnumericColAA+%
	\gnumericColAB+%
	\gnumericColAC+%
	\tabcolsep*2*3}|}%
	{\gnumericPB{\centering}\textbf{MAR}}
	&\multicolumn{4}{p{	\gnumericColAD+%
	\gnumericColAE+%
	\gnumericColAF+%
	\gnumericColAG+%
	\tabcolsep*2*3}|}%
	{\gnumericPB{\centering}\textbf{ABR}}
	&\multicolumn{4}{p{	\gnumericColAH+%
	\gnumericColAI+%
	\gnumericColAJ+%
	\gnumericColAK+%
	\tabcolsep*2*3}|}%
	{\gnumericPB{\centering}\textbf{MAI}}
\\
\hhline{|~------------------------------------|}
	\multicolumn{1}{|p{\gnumericColA}|}%
	{}
	&\multicolumn{1}{p{\gnumericColB}|}%
	{\gnumericPB{\centering}\textbf{1}}
	&\multicolumn{1}{p{\gnumericColC}|}%
	{\gnumericPB{\centering}\textbf{2}}
	&\multicolumn{1}{p{\gnumericColD}|}%
	{\gnumericPB{\centering}\textbf{3}}
	&\multicolumn{1}{p{\gnumericColE}|}%
	{\gnumericPB{\centering}\textbf{4}}
	&\multicolumn{1}{p{\gnumericColF}|}%
	{\gnumericPB{\centering}\textbf{1}}
	&\multicolumn{1}{p{\gnumericColG}|}%
	{\gnumericPB{\centering}\textbf{2}}
	&\multicolumn{1}{p{\gnumericColH}|}%
	{\gnumericPB{\centering}\textbf{3}}
	&\multicolumn{1}{p{\gnumericColI}|}%
	{\gnumericPB{\centering}\textbf{4}}
	&\multicolumn{1}{p{\gnumericColJ}|}%
	{\gnumericPB{\centering}\textbf{1}}
	&\multicolumn{1}{p{\gnumericColK}|}%
	{\gnumericPB{\centering}\textbf{2}}
	&\multicolumn{1}{p{\gnumericColL}|}%
	{\gnumericPB{\centering}\textbf{3}}
	&\multicolumn{1}{p{\gnumericColM}|}%
	{\gnumericPB{\centering}\textbf{4}}
	&\multicolumn{1}{p{\gnumericColN}|}%
	{\gnumericPB{\centering}\textbf{1}}
	&\multicolumn{1}{p{\gnumericColO}|}%
	{\gnumericPB{\centering}\textbf{2}}
	&\multicolumn{1}{p{\gnumericColP}|}%
	{\gnumericPB{\centering}\textbf{3}}
	&\multicolumn{1}{p{\gnumericColQ}|}%
	{\gnumericPB{\centering}\textbf{4}}
	&\multicolumn{1}{p{\gnumericColR}|}%
	{\gnumericPB{\centering}\textbf{1}}
	&\multicolumn{1}{p{\gnumericColS}|}%
	{\gnumericPB{\centering}\textbf{2}}
	&\multicolumn{1}{p{\gnumericColT}|}%
	{\gnumericPB{\centering}\textbf{3}}
	&\multicolumn{1}{p{\gnumericColU}|}%
	{\gnumericPB{\centering}\textbf{4}}
	&\multicolumn{1}{p{\gnumericColV}|}%
	{\gnumericPB{\centering}\textbf{1}}
	&\multicolumn{1}{p{\gnumericColW}|}%
	{\gnumericPB{\centering}\textbf{2}}
	&\multicolumn{1}{p{\gnumericColX}|}%
	{\gnumericPB{\centering}\textbf{3}}
	&\multicolumn{1}{p{\gnumericColY}|}%
	{\gnumericPB{\centering}\textbf{4}}
	&\multicolumn{1}{p{\gnumericColZ}|}%
	{\gnumericPB{\centering}\textbf{1}}
	&\multicolumn{1}{p{\gnumericColAA}|}%
	{\gnumericPB{\centering}\textbf{2}}
	&\multicolumn{1}{p{\gnumericColAB}|}%
	{\gnumericPB{\centering}\textbf{3}}
	&\multicolumn{1}{p{\gnumericColAC}|}%
	{\gnumericPB{\centering}\textbf{4}}
	&\multicolumn{1}{p{\gnumericColAD}|}%
	{\gnumericPB{\centering}\textbf{1}}
	&\multicolumn{1}{p{\gnumericColAE}|}%
	{\gnumericPB{\centering}\textbf{2}}
	&\multicolumn{1}{p{\gnumericColAF}|}%
	{\gnumericPB{\centering}\textbf{3}}
	&\multicolumn{1}{p{\gnumericColAG}|}%
	{\gnumericPB{\centering}\textbf{4}}
	&\multicolumn{1}{p{\gnumericColAH}|}%
	{\gnumericPB{\centering}\textbf{1}}
	&\multicolumn{1}{p{\gnumericColAI}|}%
	{\gnumericPB{\centering}\textbf{2}}
	&\multicolumn{1}{p{\gnumericColAJ}|}%
	{\gnumericPB{\centering}\textbf{3}}
	&\multicolumn{1}{p{\gnumericColAK}|}%
	{\gnumericPB{\centering}\textbf{4}}
\\
\hhline{|-------------------------------------|}
	 \multicolumn{1}{|p{\gnumericColA}|}%
	{\gnumericPB{\raggedright}Levantamento de literatura}
	&\multicolumn{1}{p{\gnumericColB}|}%
	{\cellcolor{black!80}}
	&\multicolumn{1}{p{\gnumericColC}|}%
	{\cellcolor{black!80}}
	&\multicolumn{1}{p{\gnumericColD}|}%
	{}
	&\multicolumn{1}{p{\gnumericColE}|}%
	{}
	&\multicolumn{1}{p{\gnumericColF}|}%
	{}
	&\multicolumn{1}{p{\gnumericColG}|}%
	{}
	&\multicolumn{1}{p{\gnumericColH}|}%
	{}
	&\multicolumn{1}{p{\gnumericColI}|}%
	{}
	&\multicolumn{1}{p{\gnumericColJ}|}%
	{}
	&\multicolumn{1}{p{\gnumericColK}|}%
	{}
	&\multicolumn{1}{p{\gnumericColL}|}%
	{}
	&\multicolumn{1}{p{\gnumericColM}|}%
	{}
	&\multicolumn{1}{p{\gnumericColN}|}%
	{}
	&\multicolumn{1}{p{\gnumericColO}|}%
	{}
	&\multicolumn{1}{p{\gnumericColP}|}%
	{}
	&\multicolumn{1}{p{\gnumericColQ}|}%
	{}
	&\multicolumn{1}{p{\gnumericColR}|}%
	{}
	&\multicolumn{1}{p{\gnumericColS}|}%
	{}
	&\multicolumn{1}{p{\gnumericColT}|}%
	{}
	&\multicolumn{1}{p{\gnumericColU}|}%
	{}
	&\multicolumn{1}{p{\gnumericColV}|}%
	{}
	&\multicolumn{1}{p{\gnumericColW}|}%
	{}
	&\multicolumn{1}{p{\gnumericColX}|}%
	{}
	&\multicolumn{1}{p{\gnumericColY}|}%
	{}
	&\multicolumn{1}{p{\gnumericColZ}|}%
	{}
	&\multicolumn{1}{p{\gnumericColAA}|}%
	{}
	&\multicolumn{1}{p{\gnumericColAB}|}%
	{}
	&\multicolumn{1}{p{\gnumericColAC}|}%
	{}
	&\multicolumn{1}{p{\gnumericColAD}|}%
	{}
	&\multicolumn{1}{p{\gnumericColAE}|}%
	{}
	&\multicolumn{1}{p{\gnumericColAF}|}%
	{}
	&\multicolumn{1}{p{\gnumericColAG}|}%
	{}
	&\multicolumn{1}{p{\gnumericColAH}|}%
	{}
	&\multicolumn{1}{p{\gnumericColAI}|}%
	{}
	&\multicolumn{1}{p{\gnumericColAJ}|}%
	{}
	&\multicolumn{1}{p{\gnumericColAK}|}%
	{}
\\
\hhline{|-------------------------------------|}
	 \multicolumn{1}{|p{\gnumericColA}|}%
	{\gnumericPB{\raggedright}Escolha do professor orientador}
	&\multicolumn{1}{p{\gnumericColB}|}%
	{\cellcolor{black!80}}
	&\multicolumn{1}{p{\gnumericColC}|}%
	{\cellcolor{black!80}}
	&\multicolumn{1}{p{\gnumericColD}|}%
	{}
	&\multicolumn{1}{p{\gnumericColE}|}%
	{}
	&\multicolumn{1}{p{\gnumericColF}|}%
	{}
	&\multicolumn{1}{p{\gnumericColG}|}%
	{}
	&\multicolumn{1}{p{\gnumericColH}|}%
	{}
	&\multicolumn{1}{p{\gnumericColI}|}%
	{}
	&\multicolumn{1}{p{\gnumericColJ}|}%
	{}
	&\multicolumn{1}{p{\gnumericColK}|}%
	{}
	&\multicolumn{1}{p{\gnumericColL}|}%
	{}
	&\multicolumn{1}{p{\gnumericColM}|}%
	{}
	&\multicolumn{1}{p{\gnumericColN}|}%
	{}
	&\multicolumn{1}{p{\gnumericColO}|}%
	{}
	&\multicolumn{1}{p{\gnumericColP}|}%
	{}
	&\multicolumn{1}{p{\gnumericColQ}|}%
	{}
	&\multicolumn{1}{p{\gnumericColR}|}%
	{}
	&\multicolumn{1}{p{\gnumericColS}|}%
	{}
	&\multicolumn{1}{p{\gnumericColT}|}%
	{}
	&\multicolumn{1}{p{\gnumericColU}|}%
	{}
	&\multicolumn{1}{p{\gnumericColV}|}%
	{}
	&\multicolumn{1}{p{\gnumericColW}|}%
	{}
	&\multicolumn{1}{p{\gnumericColX}|}%
	{}
	&\multicolumn{1}{p{\gnumericColY}|}%
	{}
	&\multicolumn{1}{p{\gnumericColZ}|}%
	{}
	&\multicolumn{1}{p{\gnumericColAA}|}%
	{}
	&\multicolumn{1}{p{\gnumericColAB}|}%
	{}
	&\multicolumn{1}{p{\gnumericColAC}|}%
	{}
	&\multicolumn{1}{p{\gnumericColAD}|}%
	{}
	&\multicolumn{1}{p{\gnumericColAE}|}%
	{}
	&\multicolumn{1}{p{\gnumericColAF}|}%
	{}
	&\multicolumn{1}{p{\gnumericColAG}|}%
	{}
	&\multicolumn{1}{p{\gnumericColAH}|}%
	{}
	&\multicolumn{1}{p{\gnumericColAI}|}%
	{}
	&\multicolumn{1}{p{\gnumericColAJ}|}%
	{}
	&\multicolumn{1}{p{\gnumericColAK}|}%
	{}
\\
\hhline{|-------------------------------------|}
	 \multicolumn{1}{|p{\gnumericColA}|}%
	{\gnumericPB{\raggedright}Definição do tema com o orientador}
	&\multicolumn{1}{p{\gnumericColB}|}%
	{\cellcolor{black!80}}
	&\multicolumn{1}{p{\gnumericColC}|}%
	{\cellcolor{black!80}}
	&\multicolumn{1}{p{\gnumericColD}|}%
	{\cellcolor{black!80}}
	&\multicolumn{1}{p{\gnumericColE}|}%
	{\cellcolor{black!80}}
	&\multicolumn{1}{p{\gnumericColF}|}%
	{}
	&\multicolumn{1}{p{\gnumericColG}|}%
	{}
	&\multicolumn{1}{p{\gnumericColH}|}%
	{}
	&\multicolumn{1}{p{\gnumericColI}|}%
	{}
	&\multicolumn{1}{p{\gnumericColJ}|}%
	{}
	&\multicolumn{1}{p{\gnumericColK}|}%
	{}
	&\multicolumn{1}{p{\gnumericColL}|}%
	{}
	&\multicolumn{1}{p{\gnumericColM}|}%
	{}
	&\multicolumn{1}{p{\gnumericColN}|}%
	{}
	&\multicolumn{1}{p{\gnumericColO}|}%
	{}
	&\multicolumn{1}{p{\gnumericColP}|}%
	{}
	&\multicolumn{1}{p{\gnumericColQ}|}%
	{}
	&\multicolumn{1}{p{\gnumericColR}|}%
	{}
	&\multicolumn{1}{p{\gnumericColS}|}%
	{}
	&\multicolumn{1}{p{\gnumericColT}|}%
	{}
	&\multicolumn{1}{p{\gnumericColU}|}%
	{}
	&\multicolumn{1}{p{\gnumericColV}|}%
	{}
	&\multicolumn{1}{p{\gnumericColW}|}%
	{}
	&\multicolumn{1}{p{\gnumericColX}|}%
	{}
	&\multicolumn{1}{p{\gnumericColY}|}%
	{}
	&\multicolumn{1}{p{\gnumericColZ}|}%
	{}
	&\multicolumn{1}{p{\gnumericColAA}|}%
	{}
	&\multicolumn{1}{p{\gnumericColAB}|}%
	{}
	&\multicolumn{1}{p{\gnumericColAC}|}%
	{}
	&\multicolumn{1}{p{\gnumericColAD}|}%
	{}
	&\multicolumn{1}{p{\gnumericColAE}|}%
	{}
	&\multicolumn{1}{p{\gnumericColAF}|}%
	{}
	&\multicolumn{1}{p{\gnumericColAG}|}%
	{}
	&\multicolumn{1}{p{\gnumericColAH}|}%
	{}
	&\multicolumn{1}{p{\gnumericColAI}|}%
	{}
	&\multicolumn{1}{p{\gnumericColAJ}|}%
	{}
	&\multicolumn{1}{p{\gnumericColAK}|}%
	{}
\\
\hhline{|-------------------------------------|}
	 \multicolumn{1}{|p{\gnumericColA}|}%
	{\gnumericPB{\raggedright}Elaboração do Projeto de Pesquisa}
	&\multicolumn{1}{p{\gnumericColB}|}%
	{}
	&\multicolumn{1}{p{\gnumericColC}|}%
	{}
	&\multicolumn{1}{p{\gnumericColD}|}%
	{\cellcolor{black!80}}
	&\multicolumn{1}{p{\gnumericColE}|}%
	{\cellcolor{black!80}}
	&\multicolumn{1}{p{\gnumericColF}|}%
	{\cellcolor{black!80}}
	&\multicolumn{1}{p{\gnumericColG}|}%
	{\cellcolor{black!80}}
	&\multicolumn{1}{p{\gnumericColH}|}%
	{\cellcolor{black!80}}
	&\multicolumn{1}{p{\gnumericColI}|}%
	{\cellcolor{black!80}}
	&\multicolumn{1}{p{\gnumericColJ}|}%
	{\cellcolor{black!80}}
	&\multicolumn{1}{p{\gnumericColK}|}%
	{\cellcolor{black!80}}
	&\multicolumn{1}{p{\gnumericColL}|}%
	{}
	&\multicolumn{1}{p{\gnumericColM}|}%
	{}
	&\multicolumn{1}{p{\gnumericColN}|}%
	{}
	&\multicolumn{1}{p{\gnumericColO}|}%
	{}
	&\multicolumn{1}{p{\gnumericColP}|}%
	{}
	&\multicolumn{1}{p{\gnumericColQ}|}%
	{}
	&\multicolumn{1}{p{\gnumericColR}|}%
	{}
	&\multicolumn{1}{p{\gnumericColS}|}%
	{}
	&\multicolumn{1}{p{\gnumericColT}|}%
	{}
	&\multicolumn{1}{p{\gnumericColU}|}%
	{}
	&\multicolumn{1}{p{\gnumericColV}|}%
	{}
	&\multicolumn{1}{p{\gnumericColW}|}%
	{}
	&\multicolumn{1}{p{\gnumericColX}|}%
	{}
	&\multicolumn{1}{p{\gnumericColY}|}%
	{}
	&\multicolumn{1}{p{\gnumericColZ}|}%
	{}
	&\multicolumn{1}{p{\gnumericColAA}|}%
	{}
	&\multicolumn{1}{p{\gnumericColAB}|}%
	{}
	&\multicolumn{1}{p{\gnumericColAC}|}%
	{}
	&\multicolumn{1}{p{\gnumericColAD}|}%
	{}
	&\multicolumn{1}{p{\gnumericColAE}|}%
	{}
	&\multicolumn{1}{p{\gnumericColAF}|}%
	{}
	&\multicolumn{1}{p{\gnumericColAG}|}%
	{}
	&\multicolumn{1}{p{\gnumericColAH}|}%
	{}
	&\multicolumn{1}{p{\gnumericColAI}|}%
	{}
	&\multicolumn{1}{p{\gnumericColAJ}|}%
	{}
	&\multicolumn{1}{p{\gnumericColAK}|}%
	{}
\\
\hhline{|-------------------------------------|}
	 \multicolumn{1}{|p{\gnumericColA}|}%
	{\gnumericPB{\raggedright}Versão final do Projeto de Pesquisa}
	&\multicolumn{1}{p{\gnumericColB}|}%
	{}
	&\multicolumn{1}{p{\gnumericColC}|}%
	{}
	&\multicolumn{1}{p{\gnumericColD}|}%
	{}
	&\multicolumn{1}{p{\gnumericColE}|}%
	{}
	&\multicolumn{1}{p{\gnumericColF}|}%
	{}
	&\multicolumn{1}{p{\gnumericColG}|}%
	{}
	&\multicolumn{1}{p{\gnumericColH}|}%
	{}
	&\multicolumn{1}{p{\gnumericColI}|}%
	{}
	&\multicolumn{1}{p{\gnumericColJ}|}%
	{}
	&\multicolumn{1}{p{\gnumericColK}|}%
	{\cellcolor{black!80}}
	&\multicolumn{1}{p{\gnumericColL}|}%
	{}
	&\multicolumn{1}{p{\gnumericColM}|}%
	{}
	&\multicolumn{1}{p{\gnumericColN}|}%
	{}
	&\multicolumn{1}{p{\gnumericColO}|}%
	{}
	&\multicolumn{1}{p{\gnumericColP}|}%
	{}
	&\multicolumn{1}{p{\gnumericColQ}|}%
	{}
	&\multicolumn{1}{p{\gnumericColR}|}%
	{}
	&\multicolumn{1}{p{\gnumericColS}|}%
	{}
	&\multicolumn{1}{p{\gnumericColT}|}%
	{}
	&\multicolumn{1}{p{\gnumericColU}|}%
	{}
	&\multicolumn{1}{p{\gnumericColV}|}%
	{}
	&\multicolumn{1}{p{\gnumericColW}|}%
	{}
	&\multicolumn{1}{p{\gnumericColX}|}%
	{}
	&\multicolumn{1}{p{\gnumericColY}|}%
	{}
	&\multicolumn{1}{p{\gnumericColZ}|}%
	{}
	&\multicolumn{1}{p{\gnumericColAA}|}%
	{}
	&\multicolumn{1}{p{\gnumericColAB}|}%
	{}
	&\multicolumn{1}{p{\gnumericColAC}|}%
	{}
	&\multicolumn{1}{p{\gnumericColAD}|}%
	{}
	&\multicolumn{1}{p{\gnumericColAE}|}%
	{}
	&\multicolumn{1}{p{\gnumericColAF}|}%
	{}
	&\multicolumn{1}{p{\gnumericColAG}|}%
	{}
	&\multicolumn{1}{p{\gnumericColAH}|}%
	{}
	&\multicolumn{1}{p{\gnumericColAI}|}%
	{}
	&\multicolumn{1}{p{\gnumericColAJ}|}%
	{}
	&\multicolumn{1}{p{\gnumericColAK}|}%
	{}
\\
\hhline{|-------------------------------------|}
	 \multicolumn{1}{|p{\gnumericColA}|}%
	{\gnumericPB{\raggedright}Estudo da medição de cargas eletroestáticas}
	&\multicolumn{1}{p{\gnumericColB}|}%
	{} &\multicolumn{1}{p{\gnumericColC}|}%
	{}
	&\multicolumn{1}{p{\gnumericColD}|}%
	{}
	&\multicolumn{1}{p{\gnumericColE}|}%
	{}
	&\multicolumn{1}{p{\gnumericColF}|}%
	{}
	&\multicolumn{1}{p{\gnumericColG}|}%
	{}
	&\multicolumn{1}{p{\gnumericColH}|}%
	{}
	&\multicolumn{1}{p{\gnumericColI}|}%
	{}
	&\multicolumn{1}{p{\gnumericColJ}|}%
	{}
	&\multicolumn{1}{p{\gnumericColK}|}%
	{}
	&\multicolumn{1}{p{\gnumericColL}|}%
	{\cellcolor{black!80}}
	&\multicolumn{1}{p{\gnumericColM}|}%
	{\cellcolor{black!80}}
	&\multicolumn{1}{p{\gnumericColN}|}%
	{}
	&\multicolumn{1}{p{\gnumericColO}|}%
	{}
	&\multicolumn{1}{p{\gnumericColP}|}%
	{}
	&\multicolumn{1}{p{\gnumericColQ}|}%
	{}
	&\multicolumn{1}{p{\gnumericColR}|}%
	{}
	&\multicolumn{1}{p{\gnumericColS}|}%
	{}
	&\multicolumn{1}{p{\gnumericColT}|}%
	{}
	&\multicolumn{1}{p{\gnumericColU}|}%
	{}
	&\multicolumn{1}{p{\gnumericColV}|}%
	{}
	&\multicolumn{1}{p{\gnumericColW}|}%
	{}
	&\multicolumn{1}{p{\gnumericColX}|}%
	{}
	&\multicolumn{1}{p{\gnumericColY}|}%
	{}
	&\multicolumn{1}{p{\gnumericColZ}|}%
	{}
	&\multicolumn{1}{p{\gnumericColAA}|}%
	{}
	&\multicolumn{1}{p{\gnumericColAB}|}%
	{}
	&\multicolumn{1}{p{\gnumericColAC}|}%
	{}
	&\multicolumn{1}{p{\gnumericColAD}|}%
	{}
	&\multicolumn{1}{p{\gnumericColAE}|}%
	{}
	&\multicolumn{1}{p{\gnumericColAF}|}%
	{}
	&\multicolumn{1}{p{\gnumericColAG}|}%
	{}
	&\multicolumn{1}{p{\gnumericColAH}|}%
	{}
	&\multicolumn{1}{p{\gnumericColAI}|}%
	{}
	&\multicolumn{1}{p{\gnumericColAJ}|}%
	{}
	&\multicolumn{1}{p{\gnumericColAK}|}%
	{}
\\
\hhline{|-------------------------------------|}
	 \multicolumn{1}{|p{\gnumericColA}|}%
	{\gnumericPB{\raggedright}Criação de algoritmo de medição em linguagem C}
	&\multicolumn{1}{p{\gnumericColB}|}%
	{}
	&\multicolumn{1}{p{\gnumericColC}|}%
	{}
	&\multicolumn{1}{p{\gnumericColD}|}%
	{}
	&\multicolumn{1}{p{\gnumericColE}|}%
	{}
	&\multicolumn{1}{p{\gnumericColF}|}%
	{}
	&\multicolumn{1}{p{\gnumericColG}|}%
	{}
	&\multicolumn{1}{p{\gnumericColH}|}%
	{}
	&\multicolumn{1}{p{\gnumericColI}|}%
	{}
	&\multicolumn{1}{p{\gnumericColJ}|}%
	{}
	&\multicolumn{1}{p{\gnumericColK}|}%
	{}
	&\multicolumn{1}{p{\gnumericColL}|}%
	{}
	&\multicolumn{1}{p{\gnumericColM}|}%
	{}
	&\multicolumn{1}{p{\gnumericColN}|}%
	{\cellcolor{black!80}}
	&\multicolumn{1}{p{\gnumericColO}|}%
	{\cellcolor{black!80}}
	&\multicolumn{1}{p{\gnumericColP}|}%
	{}
	&\multicolumn{1}{p{\gnumericColQ}|}%
	{}
	&\multicolumn{1}{p{\gnumericColR}|}%
	{}
	&\multicolumn{1}{p{\gnumericColS}|}%
	{}
	&\multicolumn{1}{p{\gnumericColT}|}%
	{}
	&\multicolumn{1}{p{\gnumericColU}|}%
	{}
	&\multicolumn{1}{p{\gnumericColV}|}%
	{}
	&\multicolumn{1}{p{\gnumericColW}|}%
	{}
	&\multicolumn{1}{p{\gnumericColX}|}%
	{}
	&\multicolumn{1}{p{\gnumericColY}|}%
	{}
	&\multicolumn{1}{p{\gnumericColZ}|}%
	{}
	&\multicolumn{1}{p{\gnumericColAA}|}%
	{}
	&\multicolumn{1}{p{\gnumericColAB}|}%
	{}
	&\multicolumn{1}{p{\gnumericColAC}|}%
	{}
	&\multicolumn{1}{p{\gnumericColAD}|}%
	{}
	&\multicolumn{1}{p{\gnumericColAE}|}%
	{}
	&\multicolumn{1}{p{\gnumericColAF}|}%
	{}
	&\multicolumn{1}{p{\gnumericColAG}|}%
	{}
	&\multicolumn{1}{p{\gnumericColAH}|}%
	{}
	&\multicolumn{1}{p{\gnumericColAI}|}%
	{}
	&\multicolumn{1}{p{\gnumericColAJ}|}%
	{}
	&\multicolumn{1}{p{\gnumericColAK}|}%
	{}
\\
\hhline{|-------------------------------------|}
	 \multicolumn{1}{|p{\gnumericColA}|}%
	{\gnumericPB{\raggedright}Estudo da transmissão de dados}
	&\multicolumn{1}{p{\gnumericColB}|}%
	{}
	&\multicolumn{1}{p{\gnumericColC}|}%
	{}
	&\multicolumn{1}{p{\gnumericColD}|}%
	{}
	&\multicolumn{1}{p{\gnumericColE}|}%
	{}
	&\multicolumn{1}{p{\gnumericColF}|}%
	{}
	&\multicolumn{1}{p{\gnumericColG}|}%
	{}
	&\multicolumn{1}{p{\gnumericColH}|}%
	{}
	&\multicolumn{1}{p{\gnumericColI}|}%
	{}
	&\multicolumn{1}{p{\gnumericColJ}|}%
	{}
	&\multicolumn{1}{p{\gnumericColK}|}%
	{}
	&\multicolumn{1}{p{\gnumericColL}|}%
	{}
	&\multicolumn{1}{p{\gnumericColM}|}%
	{}
	&\multicolumn{1}{p{\gnumericColN}|}%
	{}
	&\multicolumn{1}{p{\gnumericColO}|}%
	{}
	&\multicolumn{1}{p{\gnumericColP}|}%
	{\cellcolor{black!80}}
	&\multicolumn{1}{p{\gnumericColQ}|}%
	{\cellcolor{black!80}}
	&\multicolumn{1}{p{\gnumericColR}|}%
	{}
	&\multicolumn{1}{p{\gnumericColS}|}%
	{}
	&\multicolumn{1}{p{\gnumericColT}|}%
	{}
	&\multicolumn{1}{p{\gnumericColU}|}%
	{}
	&\multicolumn{1}{p{\gnumericColV}|}%
	{}
	&\multicolumn{1}{p{\gnumericColW}|}%
	{}
	&\multicolumn{1}{p{\gnumericColX}|}%
	{}
	&\multicolumn{1}{p{\gnumericColY}|}%
	{}
	&\multicolumn{1}{p{\gnumericColZ}|}%
	{}
	&\multicolumn{1}{p{\gnumericColAA}|}%
	{}
	&\multicolumn{1}{p{\gnumericColAB}|}%
	{}
	&\multicolumn{1}{p{\gnumericColAC}|}%
	{}
	&\multicolumn{1}{p{\gnumericColAD}|}%
	{}
	&\multicolumn{1}{p{\gnumericColAE}|}%
	{}
	&\multicolumn{1}{p{\gnumericColAF}|}%
	{}
	&\multicolumn{1}{p{\gnumericColAG}|}%
	{}
	&\multicolumn{1}{p{\gnumericColAH}|}%
	{}
	&\multicolumn{1}{p{\gnumericColAI}|}%
	{}
	&\multicolumn{1}{p{\gnumericColAJ}|}%
	{}
	&\multicolumn{1}{p{\gnumericColAK}|}%
	{}
\\
\hhline{|-------------------------------------|}
	 \multicolumn{1}{|p{\gnumericColA}|}%
	{\gnumericPB{\raggedright}Projeto de Circuitos de medição}
	&\multicolumn{1}{p{\gnumericColB}|}%
{}
	&\multicolumn{1}{p{\gnumericColC}|}%
	{}
	&\multicolumn{1}{p{\gnumericColD}|}%
	{}
	&\multicolumn{1}{p{\gnumericColE}|}%
	{}
	&\multicolumn{1}{p{\gnumericColF}|}%
	{}
	&\multicolumn{1}{p{\gnumericColG}|}%
	{}
	&\multicolumn{1}{p{\gnumericColH}|}%
	{}
	&\multicolumn{1}{p{\gnumericColI}|}%
	{}
	&\multicolumn{1}{p{\gnumericColJ}|}%
	{}
	&\multicolumn{1}{p{\gnumericColK}|}%
	{}
	&\multicolumn{1}{p{\gnumericColL}|}%
	{}
	&\multicolumn{1}{p{\gnumericColM}|}%
	{}
	&\multicolumn{1}{p{\gnumericColN}|}%
	{}
	&\multicolumn{1}{p{\gnumericColO}|}%
	{}
	&\multicolumn{1}{p{\gnumericColP}|}%
	{}
	&\multicolumn{1}{p{\gnumericColQ}|}%
	{}
	&\multicolumn{1}{p{\gnumericColR}|}%
	{\cellcolor{black!80}}
	&\multicolumn{1}{p{\gnumericColS}|}%
	{\cellcolor{black!80}}
	&\multicolumn{1}{p{\gnumericColT}|}%
	{\cellcolor{black!80}}
	&\multicolumn{1}{p{\gnumericColU}|}%
	{\cellcolor{black!80}}
	&\multicolumn{1}{p{\gnumericColV}|}%
	{}
	&\multicolumn{1}{p{\gnumericColW}|}%
	{}
	&\multicolumn{1}{p{\gnumericColX}|}%
	{}
	&\multicolumn{1}{p{\gnumericColY}|}%
	{}
	&\multicolumn{1}{p{\gnumericColZ}|}%
	{}
	&\multicolumn{1}{p{\gnumericColAA}|}%
	{}
	&\multicolumn{1}{p{\gnumericColAB}|}%
	{}
	&\multicolumn{1}{p{\gnumericColAC}|}%
	{}
	&\multicolumn{1}{p{\gnumericColAD}|}%
	{}
	&\multicolumn{1}{p{\gnumericColAE}|}%
	{}
	&\multicolumn{1}{p{\gnumericColAF}|}%
	{}
	&\multicolumn{1}{p{\gnumericColAG}|}%
	{}
	&\multicolumn{1}{p{\gnumericColAH}|}%
	{}
	&\multicolumn{1}{p{\gnumericColAI}|}%
	{}
	&\multicolumn{1}{p{\gnumericColAJ}|}%
	{}
	&\multicolumn{1}{p{\gnumericColAK}|}%
	{}
\\
\hhline{|-------------------------------------|}
	 \multicolumn{1}{|p{\gnumericColA}|}%
	{\gnumericPB{\raggedright}Teste, em bancada, dos circuitos projetados}
	&\multicolumn{1}{p{\gnumericColB}|}%
	{}
	&\multicolumn{1}{p{\gnumericColC}|}%
	{}
	&\multicolumn{1}{p{\gnumericColD}|}%
	{}
	&\multicolumn{1}{p{\gnumericColE}|}%
	{}
	&\multicolumn{1}{p{\gnumericColF}|}%
	{}
	&\multicolumn{1}{p{\gnumericColG}|}%
	{}
	&\multicolumn{1}{p{\gnumericColH}|}%
	{}
	&\multicolumn{1}{p{\gnumericColI}|}%
	{}
	&\multicolumn{1}{p{\gnumericColJ}|}%
	{}
	&\multicolumn{1}{p{\gnumericColK}|}%
	{}
	&\multicolumn{1}{p{\gnumericColL}|}%
	{}
	&\multicolumn{1}{p{\gnumericColM}|}%
	{}
	&\multicolumn{1}{p{\gnumericColN}|}%
	{}
	&\multicolumn{1}{p{\gnumericColO}|}%
	{}
	&\multicolumn{1}{p{\gnumericColP}|}%
	{}
	&\multicolumn{1}{p{\gnumericColQ}|}%
	{}
	&\multicolumn{1}{p{\gnumericColR}|}%
	{}
	&\multicolumn{1}{p{\gnumericColS}|}%
	{}
	&\multicolumn{1}{p{\gnumericColT}|}%
	{\cellcolor{black!80}}
	&\multicolumn{1}{p{\gnumericColU}|}%
	{\cellcolor{black!80}}
	&\multicolumn{1}{p{\gnumericColV}|}%
	{\cellcolor{black!80}}
	&\multicolumn{1}{p{\gnumericColW}|}%
	{}
	&\multicolumn{1}{p{\gnumericColX}|}%
	{}
	&\multicolumn{1}{p{\gnumericColY}|}%
	{}
	&\multicolumn{1}{p{\gnumericColZ}|}%
	{}
	&\multicolumn{1}{p{\gnumericColAA}|}%
	{}
	&\multicolumn{1}{p{\gnumericColAB}|}%
	{}
	&\multicolumn{1}{p{\gnumericColAC}|}%
	{}
	&\multicolumn{1}{p{\gnumericColAD}|}%
	{}
	&\multicolumn{1}{p{\gnumericColAE}|}%
	{}
	&\multicolumn{1}{p{\gnumericColAF}|}%
	{}
	&\multicolumn{1}{p{\gnumericColAG}|}%
	{}
	&\multicolumn{1}{p{\gnumericColAH}|}%
	{}
	&\multicolumn{1}{p{\gnumericColAI}|}%
	{}
	&\multicolumn{1}{p{\gnumericColAJ}|}%
	{}
	&\multicolumn{1}{p{\gnumericColAK}|}%
	{}
\\
\hhline{|-------------------------------------|}
	 \multicolumn{1}{|p{\gnumericColA}|}%
	{\gnumericPB{\raggedright}Simulação de descargas eletroestáticas em laboratório}
	&\multicolumn{1}{p{\gnumericColB}|}%
	{}
	&\multicolumn{1}{p{\gnumericColC}|}%
	{}
	&\multicolumn{1}{p{\gnumericColD}|}%
	{}
	&\multicolumn{1}{p{\gnumericColE}|}%
	{}
	&\multicolumn{1}{p{\gnumericColF}|}%
	{}
	&\multicolumn{1}{p{\gnumericColG}|}%
	{}
	&\multicolumn{1}{p{\gnumericColH}|}%
	{}
	&\multicolumn{1}{p{\gnumericColI}|}%
	{}
	&\multicolumn{1}{p{\gnumericColJ}|}%
	{}
	&\multicolumn{1}{p{\gnumericColK}|}%
	{}
	&\multicolumn{1}{p{\gnumericColL}|}%
	{}
	&\multicolumn{1}{p{\gnumericColM}|}%
	{}
	&\multicolumn{1}{p{\gnumericColN}|}%
	{}
	&\multicolumn{1}{p{\gnumericColO}|}%
	{}
	&\multicolumn{1}{p{\gnumericColP}|}%
	{}
	&\multicolumn{1}{p{\gnumericColQ}|}%
	{}
	&\multicolumn{1}{p{\gnumericColR}|}%
	{}
	&\multicolumn{1}{p{\gnumericColS}|}%
	{}
	&\multicolumn{1}{p{\gnumericColT}|}%
	{}
	&\multicolumn{1}{p{\gnumericColU}|}%
	{\cellcolor{black!80}}
	&\multicolumn{1}{p{\gnumericColV}|}%
	{\cellcolor{black!80}}
	&\multicolumn{1}{p{\gnumericColW}|}%
	{\cellcolor{black!80}}
	&\multicolumn{1}{p{\gnumericColX}|}%
	{}
	&\multicolumn{1}{p{\gnumericColY}|}%
	{}
	&\multicolumn{1}{p{\gnumericColZ}|}%
	{}
	&\multicolumn{1}{p{\gnumericColAA}|}%
	{}
	&\multicolumn{1}{p{\gnumericColAB}|}%
	{}
	&\multicolumn{1}{p{\gnumericColAC}|}%
	{}
	&\multicolumn{1}{p{\gnumericColAD}|}%
	{}
	&\multicolumn{1}{p{\gnumericColAE}|}%
	{}
	&\multicolumn{1}{p{\gnumericColAF}|}%
	{}
	&\multicolumn{1}{p{\gnumericColAG}|}%
	{}
	&\multicolumn{1}{p{\gnumericColAH}|}%
	{}
	&\multicolumn{1}{p{\gnumericColAI}|}%
	{}
	&\multicolumn{1}{p{\gnumericColAJ}|}%
	{}
	&\multicolumn{1}{p{\gnumericColAK}|}%
	{}
\\
\hhline{|-------------------------------------|}
	 \multicolumn{1}{|p{\gnumericColA}|}%
	{\gnumericPB{\raggedright}Confecção de protótipos funcionais}
	&\multicolumn{1}{p{\gnumericColB}|}%
	{}
	&\multicolumn{1}{p{\gnumericColC}|}%
	{}
	&\multicolumn{1}{p{\gnumericColD}|}%
	{}
	&\multicolumn{1}{p{\gnumericColE}|}%
	{}
	&\multicolumn{1}{p{\gnumericColF}|}%
	{}
	&\multicolumn{1}{p{\gnumericColG}|}%
	{}
	&\multicolumn{1}{p{\gnumericColH}|}%
	{}
	&\multicolumn{1}{p{\gnumericColI}|}%
	{}
	&\multicolumn{1}{p{\gnumericColJ}|}%
	{}
	&\multicolumn{1}{p{\gnumericColK}|}%
	{}
	&\multicolumn{1}{p{\gnumericColL}|}%
	{}
	&\multicolumn{1}{p{\gnumericColM}|}%
	{}
	&\multicolumn{1}{p{\gnumericColN}|}%
	{}
	&\multicolumn{1}{p{\gnumericColO}|}%
	{}
	&\multicolumn{1}{p{\gnumericColP}|}%
	{}
	&\multicolumn{1}{p{\gnumericColQ}|}%
	{}
	&\multicolumn{1}{p{\gnumericColR}|}%
	{}
	&\multicolumn{1}{p{\gnumericColS}|}%
	{}
	&\multicolumn{1}{p{\gnumericColT}|}%
	{}
	&\multicolumn{1}{p{\gnumericColU}|}%
	{}
	&\multicolumn{1}{p{\gnumericColV}|}%
	{\cellcolor{black!80}}
	&\multicolumn{1}{p{\gnumericColW}|}%
	{\cellcolor{black!80}}
	&\multicolumn{1}{p{\gnumericColX}|}%
	{\cellcolor{black!80}}
	&\multicolumn{1}{p{\gnumericColY}|}%
	{}
	&\multicolumn{1}{p{\gnumericColZ}|}%
	{}
	&\multicolumn{1}{p{\gnumericColAA}|}%
	{}
	&\multicolumn{1}{p{\gnumericColAB}|}%
	{}
	&\multicolumn{1}{p{\gnumericColAC}|}%
	{}
	&\multicolumn{1}{p{\gnumericColAD}|}%
	{}
	&\multicolumn{1}{p{\gnumericColAE}|}%
	{}
	&\multicolumn{1}{p{\gnumericColAF}|}%
	{}
	&\multicolumn{1}{p{\gnumericColAG}|}%
	{}
	&\multicolumn{1}{p{\gnumericColAH}|}%
	{}
	&\multicolumn{1}{p{\gnumericColAI}|}%
	{}
	&\multicolumn{1}{p{\gnumericColAJ}|}%
	{}
	&\multicolumn{1}{p{\gnumericColAK}|}%
	{}
\\
\hhline{|-------------------------------------|}
	 \multicolumn{1}{|p{\gnumericColA}|}%
	{\gnumericPB{\raggedright}Instalação e acompanhamento em equipamentos reais}
	&\multicolumn{1}{p{\gnumericColB}|}%
	{}
	&\multicolumn{1}{p{\gnumericColC}|}%
	{}
	&\multicolumn{1}{p{\gnumericColD}|}%
	{}
	&\multicolumn{1}{p{\gnumericColE}|}%
	{}
	&\multicolumn{1}{p{\gnumericColF}|}%
	{}
	&\multicolumn{1}{p{\gnumericColG}|}%
	{}
	&\multicolumn{1}{p{\gnumericColH}|}%
	{}
	&\multicolumn{1}{p{\gnumericColI}|}%
	{}
	&\multicolumn{1}{p{\gnumericColJ}|}%
	{}
	&\multicolumn{1}{p{\gnumericColK}|}%
	{}
	&\multicolumn{1}{p{\gnumericColL}|}%
	{}
	&\multicolumn{1}{p{\gnumericColM}|}%
	{}
	&\multicolumn{1}{p{\gnumericColN}|}%
	{}
	&\multicolumn{1}{p{\gnumericColO}|}%
	{}
	&\multicolumn{1}{p{\gnumericColP}|}%
	{}
	&\multicolumn{1}{p{\gnumericColQ}|}%
	{}
	&\multicolumn{1}{p{\gnumericColR}|}%
	{}
	&\multicolumn{1}{p{\gnumericColS}|}%
	{}
	&\multicolumn{1}{p{\gnumericColT}|}%
	{}
	&\multicolumn{1}{p{\gnumericColU}|}%
	{}
	&\multicolumn{1}{p{\gnumericColV}|}%
	{}
	&\multicolumn{1}{p{\gnumericColW}|}%
	{\cellcolor{black!80}}
	&\multicolumn{1}{p{\gnumericColX}|}%
	{\cellcolor{black!80}}
	&\multicolumn{1}{p{\gnumericColY}|}%
	{\cellcolor{black!80}}
	&\multicolumn{1}{p{\gnumericColZ}|}%
	{\cellcolor{black!80}}
	&\multicolumn{1}{p{\gnumericColAA}|}%
	{}
	&\multicolumn{1}{p{\gnumericColAB}|}%
	{}
	&\multicolumn{1}{p{\gnumericColAC}|}%
	{}
	&\multicolumn{1}{p{\gnumericColAD}|}%
	{}
	&\multicolumn{1}{p{\gnumericColAE}|}%
	{}
	&\multicolumn{1}{p{\gnumericColAF}|}%
	{}
	&\multicolumn{1}{p{\gnumericColAG}|}%
	{}
	&\multicolumn{1}{p{\gnumericColAH}|}%
	{}
	&\multicolumn{1}{p{\gnumericColAI}|}%
	{}
	&\multicolumn{1}{p{\gnumericColAJ}|}%
	{}
	&\multicolumn{1}{p{\gnumericColAK}|}%
	{}
\\
\hhline{|-------------------------------------|}
	 \multicolumn{1}{|p{\gnumericColA}|}%
	{\gnumericPB{\raggedright}Análise de resultados e avaliação da Hipótese}
	&\multicolumn{1}{p{\gnumericColB}|}%
	{}
	&\multicolumn{1}{p{\gnumericColC}|}%
	{}
	&\multicolumn{1}{p{\gnumericColD}|}%
	{}
	&\multicolumn{1}{p{\gnumericColE}|}%
	{}
	&\multicolumn{1}{p{\gnumericColF}|}%
	{}
	&\multicolumn{1}{p{\gnumericColG}|}%
	{}
	&\multicolumn{1}{p{\gnumericColH}|}%
	{}
	&\multicolumn{1}{p{\gnumericColI}|}%
	{}
	&\multicolumn{1}{p{\gnumericColJ}|}%
	{}
	&\multicolumn{1}{p{\gnumericColK}|}%
	{}
	&\multicolumn{1}{p{\gnumericColL}|}%
	{}
	&\multicolumn{1}{p{\gnumericColM}|}%
	{}
	&\multicolumn{1}{p{\gnumericColN}|}%
	{}
	&\multicolumn{1}{p{\gnumericColO}|}%
	{}
	&\multicolumn{1}{p{\gnumericColP}|}%
	{}
	&\multicolumn{1}{p{\gnumericColQ}|}%
	{}
	&\multicolumn{1}{p{\gnumericColR}|}%
	{}
	&\multicolumn{1}{p{\gnumericColS}|}%
	{}
	&\multicolumn{1}{p{\gnumericColT}|}%
	{}
	&\multicolumn{1}{p{\gnumericColU}|}%
	{}
	&\multicolumn{1}{p{\gnumericColV}|}%
	{}
	&\multicolumn{1}{p{\gnumericColW}|}%
	{}
	&\multicolumn{1}{p{\gnumericColX}|}%
	{}
	&\multicolumn{1}{p{\gnumericColY}|}%
	{}
	&\multicolumn{1}{p{\gnumericColZ}|}%
	{}
	&\multicolumn{1}{p{\gnumericColAA}|}%
	{\cellcolor{black!80}}
	&\multicolumn{1}{p{\gnumericColAB}|}%
	{\cellcolor{black!80}}
	&\multicolumn{1}{p{\gnumericColAC}|}%
	{}
	&\multicolumn{1}{p{\gnumericColAD}|}%
	{}
	&\multicolumn{1}{p{\gnumericColAE}|}%
	{}
	&\multicolumn{1}{p{\gnumericColAF}|}%
	{}
	&\multicolumn{1}{p{\gnumericColAG}|}%
	{}
	&\multicolumn{1}{p{\gnumericColAH}|}%
	{}
	&\multicolumn{1}{p{\gnumericColAI}|}%
	{}
	&\multicolumn{1}{p{\gnumericColAJ}|}%
	{}
	&\multicolumn{1}{p{\gnumericColAK}|}%
	{}
\\
\hhline{|-------------------------------------|}
	 \multicolumn{1}{|p{\gnumericColA}|}%
	{\gnumericPB{\raggedright}Elaboração textual da Pesquisa}
	&\multicolumn{1}{p{\gnumericColB}|}%
	{}
	&\multicolumn{1}{p{\gnumericColC}|}%
	{}
	&\multicolumn{1}{p{\gnumericColD}|}%
	{}
	&\multicolumn{1}{p{\gnumericColE}|}%
	{}
	&\multicolumn{1}{p{\gnumericColF}|}%
	{}
	&\multicolumn{1}{p{\gnumericColG}|}%
	{}
	&\multicolumn{1}{p{\gnumericColH}|}%
	{}
	&\multicolumn{1}{p{\gnumericColI}|}%
	{}
	&\multicolumn{1}{p{\gnumericColJ}|}%
	{}
	&\multicolumn{1}{p{\gnumericColK}|}%
	{}
	&\multicolumn{1}{p{\gnumericColL}|}%
	{}
	&\multicolumn{1}{p{\gnumericColM}|}%
	{}
	&\multicolumn{1}{p{\gnumericColN}|}%
	{}
	&\multicolumn{1}{p{\gnumericColO}|}%
	{}
	&\multicolumn{1}{p{\gnumericColP}|}%
	{}
	&\multicolumn{1}{p{\gnumericColQ}|}%
	{}
	&\multicolumn{1}{p{\gnumericColR}|}%
	{}
	&\multicolumn{1}{p{\gnumericColS}|}%
	{}
	&\multicolumn{1}{p{\gnumericColT}|}%
	{}
	&\multicolumn{1}{p{\gnumericColU}|}%
	{}
	&\multicolumn{1}{p{\gnumericColV}|}%
	{}
	&\multicolumn{1}{p{\gnumericColW}|}%
	{}
	&\multicolumn{1}{p{\gnumericColX}|}%
	{}
	&\multicolumn{1}{p{\gnumericColY}|}%
	{}
	&\multicolumn{1}{p{\gnumericColZ}|}%
	{}
	&\multicolumn{1}{p{\gnumericColAA}|}%
	{}
	&\multicolumn{1}{p{\gnumericColAB}|}%
	{}
	&\multicolumn{1}{p{\gnumericColAC}|}%
	{\cellcolor{black!80}}
	&\multicolumn{1}{p{\gnumericColAD}|}%
	{\cellcolor{black!80}}
	&\multicolumn{1}{p{\gnumericColAE}|}%
	{\cellcolor{black!80}}
	&\multicolumn{1}{p{\gnumericColAF}|}%
	{\cellcolor{black!80}}
	&\multicolumn{1}{p{\gnumericColAG}|}%
	{\cellcolor{black!80}}
	&\multicolumn{1}{p{\gnumericColAH}|}%
	{\cellcolor{black!80}}
	&\multicolumn{1}{p{\gnumericColAI}|}%
	{}
	&\multicolumn{1}{p{\gnumericColAJ}|}%
	{}
	&\multicolumn{1}{p{\gnumericColAK}|}%
	{}
\\
\hhline{|-------------------------------------|}
	 \multicolumn{1}{|p{\gnumericColA}|}%
	{\gnumericPB{\raggedright}Correções textuais e fechamento da versão final}
	&\multicolumn{1}{p{\gnumericColB}|}%
	{}
	&\multicolumn{1}{p{\gnumericColC}|}%
	{}
	&\multicolumn{1}{p{\gnumericColD}|}%
	{}
	&\multicolumn{1}{p{\gnumericColE}|}%
	{}
	&\multicolumn{1}{p{\gnumericColF}|}%
	{}
	&\multicolumn{1}{p{\gnumericColG}|}%
	{}
	&\multicolumn{1}{p{\gnumericColH}|}%
	{}
	&\multicolumn{1}{p{\gnumericColI}|}%
	{}
	&\multicolumn{1}{p{\gnumericColJ}|}%
	{}
	&\multicolumn{1}{p{\gnumericColK}|}%
	{}
	&\multicolumn{1}{p{\gnumericColL}|}%
	{}
	&\multicolumn{1}{p{\gnumericColM}|}%
	{}
	&\multicolumn{1}{p{\gnumericColN}|}%
	{}
	&\multicolumn{1}{p{\gnumericColO}|}%
	{}
	&\multicolumn{1}{p{\gnumericColP}|}%
	{}
	&\multicolumn{1}{p{\gnumericColQ}|}%
	{}
	&\multicolumn{1}{p{\gnumericColR}|}%
	{}
	&\multicolumn{1}{p{\gnumericColS}|}%
	{}
	&\multicolumn{1}{p{\gnumericColT}|}%
	{}
	&\multicolumn{1}{p{\gnumericColU}|}%
	{}
	&\multicolumn{1}{p{\gnumericColV}|}%
	{}
	&\multicolumn{1}{p{\gnumericColW}|}%
	{}
	&\multicolumn{1}{p{\gnumericColX}|}%
	{}
	&\multicolumn{1}{p{\gnumericColY}|}%
	{}
	&\multicolumn{1}{p{\gnumericColZ}|}%
	{}
	&\multicolumn{1}{p{\gnumericColAA}|}%
	{}
	&\multicolumn{1}{p{\gnumericColAB}|}%
	{}
	&\multicolumn{1}{p{\gnumericColAC}|}%
	{}
	&\multicolumn{1}{p{\gnumericColAD}|}%
	{}
	&\multicolumn{1}{p{\gnumericColAE}|}%
	{}
	&\multicolumn{1}{p{\gnumericColAF}|}%
	{}
	&\multicolumn{1}{p{\gnumericColAG}|}%
	{}
	&\multicolumn{1}{p{\gnumericColAH}|}%
	{}
	&\multicolumn{1}{p{\gnumericColAI}|}%
	{\cellcolor{black!80}}
	&\multicolumn{1}{p{\gnumericColAJ}|}%
	{\cellcolor{black!80}}
	&\multicolumn{1}{p{\gnumericColAK}|}%
	{}
\\
\hhline{|-------------------------------------|}
	 \multicolumn{1}{|p{\gnumericColA}|}%
	{\gnumericPB{\raggedright}Entrega da versão final da Pesquisa}
	&\multicolumn{1}{p{\gnumericColB}|}%
	{}
	&\multicolumn{1}{p{\gnumericColC}|}%
	{}
	&\multicolumn{1}{p{\gnumericColD}|}%
	{}
	&\multicolumn{1}{p{\gnumericColE}|}%
	{}
	&\multicolumn{1}{p{\gnumericColF}|}%
	{}
	&\multicolumn{1}{p{\gnumericColG}|}%
	{}
	&\multicolumn{1}{p{\gnumericColH}|}%
	{}
	&\multicolumn{1}{p{\gnumericColI}|}%
	{}
	&\multicolumn{1}{p{\gnumericColJ}|}%
	{}
	&\multicolumn{1}{p{\gnumericColK}|}%
	{}
	&\multicolumn{1}{p{\gnumericColL}|}%
	{}
	&\multicolumn{1}{p{\gnumericColM}|}%
	{}
	&\multicolumn{1}{p{\gnumericColN}|}%
	{}
	&\multicolumn{1}{p{\gnumericColO}|}%
	{}
	&\multicolumn{1}{p{\gnumericColP}|}%
	{}
	&\multicolumn{1}{p{\gnumericColQ}|}%
	{}
	&\multicolumn{1}{p{\gnumericColR}|}%
	{}
	&\multicolumn{1}{p{\gnumericColS}|}%
	{}
	&\multicolumn{1}{p{\gnumericColT}|}%
	{}
	&\multicolumn{1}{p{\gnumericColU}|}%
	{}
	&\multicolumn{1}{p{\gnumericColV}|}%
	{}
	&\multicolumn{1}{p{\gnumericColW}|}%
	{}
	&\multicolumn{1}{p{\gnumericColX}|}%
	{}
	&\multicolumn{1}{p{\gnumericColY}|}%
	{}
	&\multicolumn{1}{p{\gnumericColZ}|}%
	{}
	&\multicolumn{1}{p{\gnumericColAA}|}%
	{}
	&\multicolumn{1}{p{\gnumericColAB}|}%
	{}
	&\multicolumn{1}{p{\gnumericColAC}|}%
	{}
	&\multicolumn{1}{p{\gnumericColAD}|}%
	{}
	&\multicolumn{1}{p{\gnumericColAE}|}%
	{}
	&\multicolumn{1}{p{\gnumericColAF}|}%
	{}
	&\multicolumn{1}{p{\gnumericColAG}|}%
	{}
	&\multicolumn{1}{p{\gnumericColAH}|}%
	{}
	&\multicolumn{1}{p{\gnumericColAI}|}%
	{}
	&\multicolumn{1}{p{\gnumericColAJ}|}%
	{}
	&\multicolumn{1}{p{\gnumericColAK}|}%
	{\cellcolor{black!80}}
\\
\hhline{|-|-|-|-|-|-|-|-|-|-|-|-|-|-|-|-|-|-|-|-|-|-|-|-|-|-|-|-|-|-|-|-|-|-|-|-|-|}

\end{longtable}

\ifthenelse{\isundefined{\languageshorthands}}{}{\languageshorthands{\languagename}}
\endgroup
\gnumericTableEnd


\end{landscape}

% ----------------------------------------------------------
% Referências bibliográficas
% ----------------------------------------------------------
% Uncomment the following two lines if you want to have a bibliography. Please do not forget to add an entry to your bibliography and reference it by using the \cite{} command
\newpage
\vspace*{2.3cm}
%\section{REFERÊNCIAS}
\renewcommand{\bibname}{REFERÊNCIAS}
\bibliography{references}


% ----------------------------------------------------------
% Glossário
% ----------------------------------------------------------
%
% Consulte o manual da classe abntex2 para orientações sobre o glossário.
%
%\glossary

% ----------------------------------------------------------
% Apêndices
% ----------------------------------------------------------


% ----------------------------------------------------------
% Anexos
% ----------------------------------------------------------


%---------------------------------------------------------------------
% INDICE REMISSIVO
%---------------------------------------------------------------------

\end{document}
